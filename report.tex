\documentclass[]{report}

\usepackage[square, numbers]{natbib}
\usepackage{graphicx}
\usepackage{caption}
\usepackage{subcaption}
\usepackage[english]{babel}
\usepackage{enumitem}
\usepackage{array}
\usepackage{lscape}


% Title Page
\title{Software Design Study\\Final Report}
\author{Jack Browne, 1236782\\Thomas Clarke, 1162927\\Keiran Crook, xxxxxxx}


\begin{document}
\maketitle

\tableofcontents

\chapter{Introduction}
\section{Problem Definition}
%Rest of page on why we chose this problem space
%Key statistics
We began this project by exploring the 'Safety in Cycling' problem space. Our motivation for looking at this space was that we felt cyclists were involved in too many avoidable accidents. 
%Could do with a statistic here
A key part of exploring this space was speaking to different stakeholders to get their opinions on the main causes of accidents. We also looked a lot of statistics from various sources in an attempt to try and spot some interesting trends.

%How we refined the space to get our final problem definition
Initially we looked into all different formats of cycling; mountain biking, BMXing, road cycling, etc. however the data showed quite clearly that the majority of serious accidents were happening on the road. Moreover they were happening on fast moving roads during commuting hours.

%Our refined problem definition
This information allowed us to narrow the problem definition down to the following:
\begin{itemize}
  \item Reducing number involving cyclists on UK roads
  \item 
\end{itemize}
%Maybe this section can go
\section{Purpose}
This report is written for the benefit of any software or hardware engineers who are tasked with implementing this product. For this reason, in parts, the report may assume some basic technical knowledge of these fields. For example, understanding of the UML diagrams used in section ?.

The report should describe, in detail, our solution to the problem above. The detail should be enough so as that a small team of people would be able to implement the solution in its entirety with relative ease. The report should address all of they key areas of the solution and be void of any ambiguities so that it alone can clarify any queries about the product.

As well as this, the report will show the progression of our ideas and the steps taken to reach our final design. Throughout the design of this product we reached points where we had to make a justified decision from a range of possibilities. We always tried to make a rational decision based on the information available coupled with our in-depth knowledge of the problem space. Each of these key decisions will be described in detail for the benefit of the reader.
\section{Scope}

\section{Overview}
Firstly we will give an overview of the system as a whole before further explaining the key design decisions that were made. Along with this we will 
\section{Definitions, Acronyms and Abbreviations}
\begin{description}
\item[RTA] Road Traffic Accident
\item[UCI] Union Cycliste Internationale
\end{description}
\section{References}
The design decisions that we made are based on our earlier research and we will continually refer to this research throughout the report, all of which is documented in the Research Report found in the appendix. 

All other papers, articles, etc. that were used to formulate our ideas are referenced in the Bibliography.

\chapter{System Overview}
% 2 or 3 pages on what the finished syetem does
% Intended users
% Key features and why
% Rejected features and why
In line with the 'Double Diamond' model that we followed this section will describe the overall system as it was when we reached the mid-point of the two diamonds. Having done considerable amount of research into the problem space we had arrived at a range of possible solutions, or parts of solutions, and we had to figure out which of these were the best\ldots





\section{Rejected features}
During our research we considered a multitude of different approaches to improving safety in cycling and as with any project we had to reject some of these features when we came up with a plan for our system. because they were either unnecessary, 



\paragraph{Wearable tech}As a group we very quickly latched onto the idea of incorporating wearable technology into our final solution - particularly inspired by recently evolving wearable eyewear technology such as Google Glass. We felt that one of the problems we would have is that a bike has no significant interface/dashboard and wearable eyewear is a very natural solution to this problem. The other interface solutions that we came across were: 
\begin{itemize}
  \item a smartphone mounted on the handlebars
  \item a small on-board-computer
  \item more recently, smartwatches
\end{itemize}
However all of these have the same major problem; they require the cyclist to divert their attention away from the road. Lack of attention, from either party, is one of main causes of RTAs involving bicycles. %statistic!
The wearable tech However this feature was eventually rejected following a peer-review. Several people failed to grasp the concept

\chapter{Design Considerations}
\section{Assumptions}
% 2 pages on the above
% Assumptions about users (age, race, gender, disability)
% Assumptions about environment (road, many other users, etc.)
% Assumptions about interface (All users will have a phone)
% Constraints (Cost, weight/size)
When designing this system we have had to make a few assumptions about certain characteristics of the end users, this is because it would be almost impossible to design a system of this nature that is both effective and easy for everybody to use. 

Firstly we have assumed that the end user will be fully able-bodied. We feel that this is a necessary assumption despite the fact that one in twenty cycling commuters are disabled \citep{census-dis}. Cycling is a very physical activity and generally those with severe physical disabilities are better off with a bike that is specifically designed to cater for their particular disability. It is very much possible that our final design will be usable for the less severely disabled, however we will not be creating our design with any of these disabilities in mind. 

Secondly we will be assuming that end users have some familiarity of bikes, particularly road bikes. By this we mean that we will not be designing an educational tool that teaches people how to ride a bike nor an assistance tool that aids those who cannot ride a bike. This is a reasonable assumption because it would be against recommended practice for a non-cyclist to start out by cycling on main roads.
%Who's recommended practice? Sources?

In fitting with both of the above assumptions, we will also be assuming that the bike is to be used by fully grown adults. As well as the fact that the bike

\section{Constraints}
There are some overall restrictions that the product naturally has in order for it to be feasible or usable. The main two that we considered were:
\begin{itemize}
  \item \textbf{Cost} because in order for a solution to be considered suitable it must be available to the general population of commuting cyclists. A solution that does everything that we set out to achieve but could never be purchased by even the richest of cyclists could not be considered suitable.
  \item \textbf{Weight} because this has a significant impact on whether the bike is actually ridable. We want to design a solution for which it is, not only possible but, comfortable to ride on the majority of commuting roads. Some of which will involve steep inclines.  
\end{itemize}
\subsection{Cost}
We are attempting to create a solution for the average commuting cyclist to use on the road. This means it is important that the average commuting person is able to afford the product. To do this we must be able to find the right components and produce the bike for a reasonable cost. To decide what constitutes 'reasonable' we analysed how much cyclists currently spend on their bikes and asked cyclists how much they would be willing to spend on a bike of this style/quality.
\begin{figure}
\centering
\includegraphics[scale=0.9]{figures/cars-bikes-spending}
\caption{Graph showing the retail sales of cars and bikes \cite{uk-stats}}
\label{fig:cars_bikes_spending}
\end{figure}
\paragraph{Current spending}
Overall spending on bikes is slowly increasing at the same time as spending on cars rapidly decreasing as shown by figure \ref{fig:cars_bikes_spending}.Possibly indicating a   However, last year in the UK, the average amount a person spends on buying a bike was only £233. However, this statistic is likely skewed by the large amount of cheap bikes sold to casual cyclists; these are not the people we are aiming this product towards\cite{spending-more}. There are no readily available statistics that differentiate between the spending on road bikes and otherwise but we were able to get a fairly good idea using statistics from the Cycle to Work scheme - which allows people to buy commuting bikes and equipment tax free. This saw commuters spending an average of over £1,100\cite{spending-more}. This is more in line with the price our bike would retail for and the cyclists we are aiming at. As well as this, our earlier research found that commuters spend a further £195 on accessories alone, many of which may be incorporated directly into our design.

\begin{figure}
\centering
\includegraphics[scale=0.8]{figures/cycle-theft}
\caption{Graph showing the thefts of vehicles and bikes \cite{ctc-stats}}
\label{fig:theft_graph}
\end{figure}
\paragraph{Concerns}
As can be expected; cyclists are always concerned about the theft of their bike. Strangely, thefts of or from cars and other vehicles has been steadily decreasing since the late 90s. While in that time thefts of bicycles has generally increased (Fig. \ref{fig:theft_graph}). This is a worrying trend and may be a contributing factor to people opting for cheaper bikes.

However, we can reasonably assume that consumers who spend more on their products will be  willing to spend a little extra on security or insurance. This might go some way to counteracting the concern about theft. Another way to alleviate these concerns might be to incorporate anti-theft devices into the product itself, similarly to ‘Find my iPhone’ and similar products.
\paragraph{Conclusion}
From the research we decided that an appropriate cost for producing this bike is \pounds1000. Even though this is significantly more than the average spend for a bike we feel that it is justified for a product of this quality since it is in the region that cycling commuters currently spend (particularly once accessories are taken into account). Also, this still comes at considerably less cost than many of the commuting alternatives (diriving, public transport,etc).

\subsection{Weight}
There are not many statistics out there regarding the weight of consumer bicycles so most of this research came from talking to cyclists and cycle shops.
\paragraph{Current products}
UCI impose a minimum weight of 15 lbs, for this reason many cyclists consider anything around this weight or lighter to be a \textit{light} bike. However, the retailers that we spoke to were generally of the opinion that bikes at this end of the scale were only really appropriate for competitive racing environments. They thought that 23 lbs was a good weight for a road bike or 30 lbs for a more general purpose mountain bike. These weights give a good trade off between speed and ease of ride.

\paragraph{Electric bikes}
Working with the 23 lb ballpark figure, as suggested by the cycling retailers that we spoke to, we noticed a problem. Given that this product heavily relies on a computer performing some calculations and some sensors feeding information into the computer we clearly will require some sort of battery. Looking at the current bike batteries on the market this would not be easy to fit into our 23 lb restriction. In the Specialized Turbo S, for example, the battery alone weighs 8 lbs\cite{turbo-s}. This would take up over $ \frac{1}{3} $ of our weight limit. This doesn't leave much room for a frame, wheels and pedals; let alone sensors and a computer. When all of this is factored in, we are realistically looking at a minimum of 45/50 lbs. This is likely to be almost to twice the weight that many cyclists are used to and as a result may be very difficult to ride. The obvious solution to this problem is to heavily assist the cyclist with a motor powering the wheels. This is the primary use case for current electric bikes so the technology already exists and shouldn't be too difficult to implement

\paragraph{Conclusion}
Cyclists have a reputation for being obsessed with weight - trying to shave off pounds at every opportunity. However our research showed that most commuters probably wouldn't be overly concerned about a slight increase in weight if it meant an equally good (or better) performance and a more comfortable ride. This makes sense because, in reality, the cyclists themselves are usually by far the biggest contributor to the weight of the bike. Also, as a result of this research, it is likely that our final product will need contain some form of motor or other electronic assistance which will help to carry some of the weight. This puts our product more in line with electric bikes on the market, rather than regular road bikes. This gives some additional flexibility regarding the weight of the bike.

In conclusion we have decided that we will design the bike with a 60 lb restriction. This should allow us to include all of the hardware components that will be necessary for our solution, with the bike still being comfortable to ride. 

\section{Design Goals and Guidelines}
% 1 page on U(X/C)D and why this is suitable for our product
% Double diamond model (Design council)
In the very initial stages of this project we looked into some different design processes

\chapter{Design Decisions}

\section{Communication}

\subsection{Wireless}

\begin{landscape}

\begin{table}[h]
\resizebox{1\textwidth}{!}{\begin{minipage}{\textwidth}

    \begin{tabular}{ | m{2.5cm} | m{2.5cm} | m{2.5cm} | m{2.5cm} | m{2.5cm} | m{2.5cm} | m{2.5cm} | m{2.5cm} |}
    \hline
    \textbf{Wireless comms} & \textbf{Max Data Rate} & \textbf{Security Protocol} & \textbf{Encryptionl} & \textbf{Connection Latency} & \textbf{Avg Power Consumption} & \textbf{Range} & \textbf{Backwards Compatible}\\ \hline
   
   Bluetooth v2.0 + EDR & 2.1 Mbit/s & & 128 bit stream cipher & & Class 1 radio: 100mW &   \\ \hline
   Bluetooth v2.1 + EDR & 3 Mbit/s & & 128 bit stream cipher & & Class 1 radio: 100mW & \\ \hline
   Bluetooth v3.0 HS & 24 Mbit/s & & 128 bit stream cipher & & & & \\ \hline

   
    \end{tabular}


\caption[Table caption text]{Wireless communication methods comparison} 
\label{table:wireless_comp}
\end{minipage} }
\end{table}

\end{landscape}


\chapter{System Architecture}
\section{Architectural Design}

\subsection{Bicycle User Access}  

\paragraph{} Based around security and convenience, users would have their own separate accounts and the bicycle would be associated with one master account. Only the master user and users, that the aforementioned has allowed, can access the bicycle and use it. This offers security and convenience as users have their own account with their preferences for the bicycle to automatically use. 

\subsubsection{\textbf{Security}}

\paragraph{Requirements}

\begin{itemize}
\item A smart phone should be able to be used a ?key? to the bicycle
\item The Bicycle should be associated with one master user and their account.
\item The master account should be able to allow other users and their accounts pair, unlock and use the bicycle

\end{itemize}


\subsubsection{Communications}


\paragraph{Internet communication}

All internet traffic between the bicycle and the user will go through the HTTPS protocol, which uses the Transport Layer Security (TLS) protocol to offer encrypted data. TLS Uses X.509 certificates (Public Key Certificate standard), which uses asymmetric cryptography, to ensure the authenticity of party and to exchange a symmetric key. The authenticity of our domain w This symmetric key can be used for the rest of the session or after a time interval. The use of X.509 certificate will require the use of  certificate authorities, as people can most people will not be able to trust the authenticity of a self signed certificate.


\chapter{Data Design}
\section{Data Description}
\section{Data Dictionary}

\chapter{Component Design}
\section{Collision Avoidance System}
Why we need this? What is the system responsible for?
\subsection{Component Requirements}
\begin{itemize}
  \item Function during times of the day when there are levels of light.
  \item Detect vehicles (differentiating them from other objects) and then apply a unique identifier to that vehicle.
  \item Track vehicles and acknowledging them with their unique identifier over time.
  \item Notify the user when there is a possible collision.
  \item The future trajectory of the vehicle should be predicted and, in addition to, the bicycles speed, direction, GPS route) used to to calculate the probability of a collision.
  \item Apply an evasive action when the algorithm predicts that a collision, involving the user, is of a high enough probability. 
\end{itemize}

\paragraph{}

We compared the 

\begin{table}[h]
\resizebox{1\textwidth}{!}{\begin{minipage}{\textwidth}

    \begin{tabular}{ | m{1.2cm} | p{4.7cm} | p{4.7cm} |}
    \hline
    \textbf{Device} & \multicolumn{1}{|c|}{\textbf{Advantages}} & \multicolumn{1}{|c|}{\textbf{Disadvantages}} \\ \hline
   
    Radar & 
    
    \begin{itemize}[leftmargin=*]
     
    	\item Sees through fog perfectly
   
    	\item A radar hit returns distance and an objects speed
    	
    \end{itemize} &
   
    \begin{itemize}[leftmargin=*]   
       	\item Current consumer radar offer low resolution
       	
       	\item Radar struggles to determine object positions
       	
       	\item Fixed objects can not be discerned from stationary vehicles
       	
       	\item Expensive in comparison to cameras  
       \end{itemize}  \\ \hline
       
    Lidar & 
    
    \begin{itemize}[leftmargin=*]   
    	\item Evaluating an objects distance is easier than with a camera
    
    	\item 3D map, meaning finding an objects position is trivial 
    
    	\item Emits light and can therefore work without the need of ambient light.
    \end{itemize} &
    
      \begin{itemize}[leftmargin=*]   
    
    	\item Low resolution 
    
    	\item Limited range; best results up to 70m
    
    	\item Moving parts - more prone to damage from movement
    
    	\item Expensive in comparison to cameras
    
	\end{itemize} \\ \hline 
      
    Colour Camera &
    
     \begin{itemize}[leftmargin=*]   
        	\item Inexpensive
        	
        	\item Due to their uses of reflected light, their range during daylight is arbitrary 
        	
        	\item Very high resolution (3000 lines, compared to a LIDARs 64)
        	
        	\item Due to their colour detection and resolution they offer future computer vision updates. e.g. reading road markings, signs, etc.
        	
        	\item No moving parts.
     \end{itemize} &
        
     \begin{itemize}[leftmargin=*]   
       	\item At night they require emitted light (headlight)
       
       \item Have to content with light variations (e.g. moving shadows)
       
       \item Emitted light at night may not generate a high enough lumens per square foot
      
       \item Computer vision requires higher processing power
        
    \end{itemize} \\ \hline 
    
   
    
    
    \end{tabular}


\caption[Table caption text]{Display the advantages and disadvantages of different hardware for object detection and tracking} 
\label{table:detect_hardware_comp}
\end{minipage} }
\end{table}



\subsection{Interface Description} What does the components interact with (other software, hardware) diagram?

\subsection{Possible Solutions} Collision has to be broken down into 4 different steps (detection, tracking ...)

\subsubsection{Preceding Vehicle Trajectory Prediction by Multi-Cue Integration \citep{multi-cue}}

\paragraph {} This is paper \citep{multi-cue} detects and tracks preceding vehicles using a monochrome camera to predict whether the vehicle is going to change lane (in a left or right direction) or stay in its current lane. This involves the detection and tracking of both vehicles and the road lines they reside in. It uses Support Vector Machine (SVM) with a motion and appearance cue to recognise driver pattens, that are trained for a two class feature set. The two class features are used to predict lane-changing recognition, via the input of timestamped 3D positioning. 

\paragraph{}A kalman filter is used to track the motion of the lane boundary and predict from frame-frame its new location. There is an intersection between two lane boundaries, which is known as the vanishing, and is used to detect the pitch angle change of the camera. Line features are used to establish vehicle hypotheses, which are verified by an SVM (with HOG) trained for vehicle recognition via a pre-learned vehicle model database. Post verification, a Kanade-Lucas-Tomasi feature tracker is used and the bottom of the tracked vehicle in each frame to calculate its 3D position.
\paragraph{Advantages}
\begin{itemize}
\item Tested on real-world data via a vehicle on a highway
\item Lane changing warning is warned in real-time
\item The whole system achieves 10hz on a 1.8 GHz Dual-Core CPU
\end{itemize}

\paragraph{Disadvantages}
\begin{itemize}
\item The system requires lanes to infer whether a collision is probable	 	
\end{itemize}

\begin{figure}[h]
\centering
	\begin{minipage}{0.49\textwidth}
		\centering
		\includegraphics[scale=0.4]{figures/research_paper_figures/trajectory_multi-cue}
		\caption{Vehicle trajectory computation at straight and curved lanes \citep{multi-cue} }
		\label{fig:traj_multi-cue_image1}
	\end{minipage}\hfill
	\begin{minipage}{0.49\textwidth}
		\centering
		\includegraphics[scale=0.2]{figures/collision_avoidance_figures/Vehicle_changing_lane_into_the_path_of_a_cyclist}
		\caption{Cyclist and vehicle lane changing collision}
		\label{fig:change_lane_image}
	\end{minipage}
\end{figure}


\paragraph{}

Fig. ~\ref{fig:traj_multi-cue_image1} on page~\pageref{fig:traj_multi-cue_image1} shows the lane change tracking and prediction that this papers algorithms accomplish and Fig. ~\ref{fig:change_lane_image} on page~\pageref{fig:change_lane_image} shows, one of other, cyclist collisions that we want to prevent. This paper displays a high chance of solving this issue.



\subsubsection{Learning Multi-Lane Trajectories using Vehicle-Based Vision \citep{multi-lane_traj}}

Conclusion - decision complying with criteria at the start

\chapter{User Interface Design}
\section{Overview of User Interface}


\paragraph{}

\section{Screen Images}
\section{Screen Objects and Actions}


\chapter{Prototype Analysis}
We created prototypes of the UI and of the bike hardware 
\section{Prototype 1}

\section{Prototype 2}

\chapter{Detailed Design}

\chapter{Conclusion}

\bibliographystyle{plainnat}
\bibliography{bibliography}

\end{document}          
