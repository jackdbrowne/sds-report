\documentclass[]{report}

\usepackage[round, authoryear]{natbib}

% Title Page
\title{Software Design Study\\Final Report}
\author{Jack Browne, 1236782\\Thomas Clarke, 1162927\\Keiran Crook, xxxxxxx}


\begin{document}
\maketitle

\tableofcontents

\chapter{Introduction}
\section{Problem Definition}
We began this project by exploring the 'Safety in Cycling' problem space. Our motivation for this was that we felt cyclists were involved in too many avoidable accidents. 
%Could do with a statistic here
A key part of exploring this space was speaking to different stakeholders to get their opinions on the main causes of accidents. We also looked a lot of statistics from various sources in an attempt to try and spot some interesting trends.

Initially we looked into all different formats of cycling; mountain biking, BMXing, road cycling, etc. however the data showed quite clearly that the majority of serious accidents were happening on the road. Moreover they were happening on fast moving roads during commuting hours.

This information allowed us to narrow the problem definition down to the following:
\begin{itemize}
  \item Reducing number involving cyclists on UK roads
  \item 
\end{itemize}
\section{Purpose}
This report is written for the benefit of any software or hardware engineers who are tasked with implementing this product. For this reason, in parts, the report may assume some basic technical knowledge of these fields. 

The report should describe, in detail, our solution to the problem above. The detail should be enough so as that a small team of people would be able to implement the solution in its entirety with relative ease. The report should address all of they key areas of the solution and be void of any ambiguities so that it alone can clarify any queries about the product.
\section{Scope}

\section{Overview}
Firstly we will give an overview of the system as a whole before further explaining the key design decisions that were made. 
\section{Definitions, Acronyms and Abbreviations}
\begin{description}
\item[RTA] Road Traffic Accident
\end{description}
\section{References}
The design decisions that we made are based on our earlier research and we will continually refer to this research throughout the report, all of which is documented in the Research Report found in the appendix. 

All other papers, articles, etc. that were used to formulate our ideas are referenced in the Bibliography.

\chapter{System Overview}

\chapter{Design Considerations}
\section{Design Assumptions, Dependencies and Constraints}
When designing this system we have had to make a few assumptions about certain characteristics of the end users, this is because it would be almost impossible to design a system of this nature that is both effective and easy for everybody to use. 

Firstly we have assumed that the end user will be fully able-bodied. We feel that this is a necessary assumption despite the fact that one in twenty cycling commuters are disabled \citep{wheelsforwell}. Cycling is a very physical activity and generally those with severe physical disabilities are better off with a bike that is specifically designed to cater for their particular disability. It is very much possible that the final product will be usable for the less severely disabled, however we will not be creating our design with any of these disabilities in mind.

Secondly we will be assuming that end users some familiarity of bikes and particularly road bikes. We will not be designing an educational tool that teaches how to ride a bike nor we will it be an assistance 
\section{Design Goals and Guidelines}
During the design process our overriding guideline

\chapter{Data Design}
\section{Data Description}
\section{Data Dictionary}

\chapter{System Architecture}
\section{Architectural Design}
\chapter{Component Design}

\section{Collision Avoidance System}

Why we need this? What is the system responsible for?

\subsection{Component Requirements}

\begin{itemize}
  \item Function during times of the day when there are levels of light.
  \item Detect vehicles (differentiating them from other objects) and then apply a unique identifier to that vehicle.
  \item Track vehicles and acknowledging them with their unique identifier over time.
  \item Notify the user when there is a possible collision.
  \item The future trajectory of the vehicle should be predicted and, in addition to, the bicycles speed, direction, GPS route) used to to calculate the probability of a collision.
  \item Apply an evasive action when the algorithm predicts that a collision, involving the user, is of a high enough probability. 
  
\end{itemize}

\subsection{Interface Description} What does the components interact with (other software, hardware) diagram?

\subsection{Possible Solutions} Collision has to be broken down into 4 different steps (detection, tracking ...)

\subsubsection{Detection}

Solution 1

\paragraph 

Adj 
Dis

Solution 2

Adj 
Dis

Conclusion - decision complying with criteria at the start

\subsubsection{Tracking}

\subsubsection{Trajectory Prediction}

\subsubsection{Evasive Action}


\chapter{User Interface Design}
\section{Overview of User Interface}
\section{Screen Images}
\section{Screen Objects and Actions}

\chapter{Detailed Design}

\chapter{Libraries and Tools}

\chapter{Time Planning}

\chapter{Conclusion}

\bibliographystyle{plainnat}
\bibliography{bibliography}

\end{document}          
